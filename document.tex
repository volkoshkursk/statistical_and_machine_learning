\documentclass[11pt]{beamer}
\usepackage[utf8]{inputenc}
\usepackage{lmodern}
\usepackage[T2A]{fontenc}
\usepackage{cmbright}
\usepackage[russian]{babel}
\usetheme{Darmstadt}
\usepackage{amsmath}
\usepackage{amsfonts}
\usepackage{bm}
% Использовать полужирное начертание для векторов
\let\vec=\mathbf

\DeclareMathOperator{\mathspan}{span}
\DeclareMathOperator{\mathdim}{dim}
\DeclareMathOperator{\rank}{rank}
\DeclareMathOperator{\diag}{diag}
\begin{document}
	\author{Е. Ларин, Ф. Ежов, И. Кононыхин }
	\title{Обучение с учителем. Классификация. Дискриминантный анализ. }
	%\subtitle{}
	%\logo{}
	\institute{Санкт-Петербургский государственный университет 
		
		Прикладная математика и информатика
		
		Вычислительная стохастика и статистические модели
	}
	\date{}
	\subject{Семинар по статистическому и машинному обучению}
	\setbeamercovered{transparent}
	\setbeamertemplate{navigation symbols}{}
	\begin{frame}[plain]
		\maketitle 
	\end{frame}
	
	\begin{frame}
		\frametitle{Обучение с учителем}
		Выборка из генеральной случайной величины
		\begin{equation}\bm{X} \in \mathbb{R}^{n\times p}, \;\;\mathbf{y}\in \mathbb{R}^n\end{equation}
		
		
	\end{frame}
	\begin{frame}
		\frametitle{Обучение с учителем}
		\begin{block}{Классификация}
			Sample text
		\end{block}
	\end{frame}
\end{document}